\documentclass[12pt]{article}
\def\x#1#2{$\mathbb{#1}^#2$} 
\def\n#1{\x#1}

\usepackage{graphicx}
\usepackage{latexsym}
\usepackage{amsfonts}
\usepackage{listings}
\usepackage{subfigure}
\usepackage{epic}


\title{ADS - Heuropt - Abgabe 1} 
\author{Kern, Weichselbaum}


\begin{document}
	
\maketitle

We implemented the second exercise with an ACO. To be more specific, we decided to use a \textbf{MAX - MIN ACO}.\\
There are several MEB implementations using ACO's, therefor we followed their advice regarding pheromones and used a \textbf{lower bound} of 0.00001 and an \textbf{upper bound} of 0.9999. The \textbf{learning rate} is adjusted (based on the convergence factor discussed below) between 0.1 and 0.15, as this seems to yield the best results.\\
\textbf{Pheromones} are \textbf{initiated} with 0.5.
\\
16 Ants are run and the ant resulting in the lowest cost is used for updating our pheromones. \\
One \textbf{ant} constructs a valid tree, starting with the root node. From that a neighborhood is calculated. The neighborhood includes all edges from all trienodes to all non-trienodes. The chosen edge is chosen based on the probability of the combination of global and local information.\\
\textbf{Alpha} and \textbf{Beta} are chosen to be 0.4 and 0.7 respectively.\\
On top of that we \textbf{sweep}. This means we take the costliest edge of each node and try to add it to another node, iff the new costs are lower than the old ones. This is done until no edge can be re-assigned, per node.\\
After the sweep, we apply an \textbf{VND}. The function used for the VND is an \textbf{r-shrink}. Since we can change the $r$ dynamically, it is our only neighborhood for the $VND$. The $r$ goes from 1 to 15. No matter how many nodes the instance has, 15 seems to be the upper bound still yielding better results.\\
The r-shrink uses every node which has equal or more than r children. It then removes those r children from the node and re-assigns them in the best possible way. If this yields a better trie than before the trie is used further and returned to the ant.\\
\\
The \textbf{convergence factor} is used to determine how saturated the pheromones are. The higher the pheromone on the edges of the currently best trie the higher the CF is. This is then used to determine whether or not a pheromone reset is due, to be more specific as soon as the CF hits 0.99 or above a reset is due. If this is the case all pheromones are set to 0.5.\\
In addition to the CF reset we have a reset if the trie doesn't change for $n$ iterations. The $n$ can be set quite liberal. If it is between 10 and 100 the difference is almost unnoticeable.
\\
The ACO updates and evaporates the pheromones. The result is stored in a trie.
\\
\\

\begin{table}
\centering
\begin{tabular}{rllll} 
  \hline
  Instance & best & avg & std & time (sec) \\
  \hline \hline
  	6 & 440 & 450 & 8 & 15 \\
  \hline
	8 & 346 & 356 & 2 & 39 \\
  \hline
	10 & 310 & 325 & 2 & 128 \\
  \hline
\end{tabular}
\label{table:gammaSSE}
\caption{Results of 100 runs}
\end{table}

\end{document}